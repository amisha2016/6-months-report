
%\section{Other Specifications}

A Software Requirements Analysis for a software system is a complete 
description of the behavior of a system to be developed. It include functional Requirements
and Software Requirements. In addition to these, the SRS contains 
non-functional requirements. Non-functional requirements are 
requirements which impose constraints on the design or implementation.
\begin{itemize}
\item{\bf Purpose}: OpenStreetMap (OSM) is an open collaborative project to  create a free editable map of the world and the main purpose of this project is to:
\begin{enumerate}
\item  To  create a free editable map of the world.
\item To gather location data  using GPS, local knowledge, and other free sources of information and upload it.
\item  To encourage the growth, development and distribution of free geospatial data. 
\item  To provide geospatial data for anyone to use and share.
\item Reduce the time for analysis.
\item The OpenStreetMap Foundation is an international not-for-profit organization supporting, but not controlling, the OpenStreetMap Project.
\end{enumerate}

\item{\bf Users of the System}
\begin{enumerate} 
\item Client : Clients are the end users that benefit from this project.
They just provide input(like entering name of the area) and gets output(in the form of map).
\begin{enumerate}
\item Researcher or student-: They have knowledge of working of procedures and can edt the map according to their needs.  
\end{enumerate}
\end{enumerate}
\end{itemize}

\subsection{Functional Requirements}
\begin{itemize}
\item {\bf Specific Requirements}: This phase covers the whole requirements 
for the system. After understanding the system we need the input data 
to the system then we watch the output and determine whether the output 
from the system is according to our requirements or not. So what we have 
to input and then what we'll get as output is given in this phase. This 
phase also describe the software and non-function requirements of the 
system.
\item {\bf Input Requirements of the System}
\begin{enumerate} 
\item Guess points and name of the places.
\item Precision
\item Required point at which value is to be found
\item Knowledge of latitude and longitude.
\end{enumerate}
\vskip 0.5cm
\item {\bf Output Requirements of the System}
\begin{enumerate} 
\item Final output of the location of the particulaar area.
\item Shops, restaurants and many more are represented through icon and images.
\end{enumerate}
\vskip 0.5cm
\item {\bf Special User Requirements}
\begin{enumerate} 
\item Taking bulk input values through html forms.
\end{enumerate}
\vskip 0.5cm
\item {\bf Software Requirements}
\begin{enumerate} 
\item Programming language: C++, Python
\item software: \LaTeX{}
\item Web Languages: php, javascript, html
\item Database: Postgresql 
\item Documentation: Doxygen 1.8.3
\item Text Editor: Vim
\item Operating System: Ubuntu 14.04 or 15.10
\item Revision System: Git

\end{enumerate}
\vskip 0.5cm
\subsection{Non functional requirements}
\begin{enumerate} 
\item Scalability: System should be able to handle a number of users. 
For e.g., handling around thousand users at the same time.
\item Usability: Simple user interfaces that a layman can understand.
\item Speed: Processing input should be done in reasonable time
 i.e. we can say maximum 24 hrs.
\end{enumerate}
\end{itemize}


