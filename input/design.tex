\section{Problem Formulation}
Geographical data (geo data) is not free in many parts of the world.\\
Because that data is copyrighted and owned by multiple organisations like the Ordnance Survey. Google/whoever just licenses it. If we were to use it, we'd have to pay for it. \\
\noindent "OpenStreetMap is a free editable map of the whole world. It is made by people like you." Which means the database will always be subject to the whims, experimentation, and mistakes of the community. This is precisely OSM's strength since, among other things, it allows our data to quickly accommodate changes in the physical world.\\
\noindent By making your system an OSM tile server not only you can edit the map but can use it offline also. You can change the styling of the map like color of the roads fonts style and amny more as per your requirments. 

\section{Feasibility Study}
\noindent Feasibility analysis involved a thorough assessment of the operational and technical aspects of the proposal.
Feasibility study tested the system proposal and identified whether the user needs may be satisfied using
the current software and hardware technologies, whether the system will be cost effective from a business
point of view and whether it can be developed with the most up to date technologies.
\subsection{Operational Feasibility}
\noindent Operational feasibility is a measure of how well a project solves the problems, and takes advantage of
the opportunities identified during scope definition and how it satisfies the requirements identified in
the requirements analysis phase of system development. All the operations performed in the software
are very quick and satisfy all the requirements.
\subsection{Technical Feasibility}
\noindent Technological feasibility is carried out to determine whether the project has the capability, in terms
of software, hardware, personnel to handle and fulfill the user requirements. The assessment is based
on an outline design of system requirements in terms of Input, Processes, Output and Procedures.
OpenStreetMap is technically feasible as it is built up using various open source technologies and it can run on any platform.
\subsection{Economic Feasibility}
\noindent Economic analysis is the most frequently used method to determine the cost/benefit factor for evalu-
ating the effectiveness of a new system. In this analysis we determine whether the benefit is gained
according to the cost invested to develop the project or not. If benefits outweigh costs, only then
the decision is made to design and implement the system. It is important to identify cost and benefit
factors, which can be categorized as follows:
\begin{itemize}
\item Development Cost
\item Operation Cost
\end{itemize}
This System is Economically feasible with 0 Development and Operating Charges
as it is developed in Qt Framework and libdxfrw library which is open source technology and is available free of cost on the internet.

\section{Facilities required for proposed work}
\subsection{Hardware Requirements}
\begin{itemize}
\item Operating System: Linux/Windows
\item Processor Speed: 512KHz or more
\item RAM: Minimum 2GB
\end{itemize}
\subsection{Software Requirements}
\begin{itemize}
\item Library: Mapnik
\item Modules: Mod\_tile
\item Compiler: CartoCSS
\item Stylesheet: OSMBright 
\item Programming Language: C++, Python 
\end{itemize}

\section{Methodology}
\begin{itemize}
\item Studying various methods available to solve different problems of numerical analysis.
\item Deciding various input and output parameters of methods.
\item Making the approach modular 
\item Styling the map.
\item Generating documentation
\end{itemize}

\section{Project Work} 
\textbf{Studied Previous System:}\\
Before starting the project, \\\\
\textbf{Learn Linux:}\\
Before starting with project, we have to install various things to make our system an OSM server. So, for that you should know terminal commands because I gonna explain it for Ubuntu only. It is possible on other OS also but you have to work it own. I have provided some basic command also for Linux. \\\\

\textbf{Learn postgresql:}\\
 We have to go through the basics of postgresql(database) also, such that there
should not be any problem proceeding with project.\\\\
\textbf{Make or Cmake}\\
The softwares like mapnik, mod\_tile, osm2pqsql, are compiled through the Cmake which is basically language. So, we should the basics of it.
\\\\
\textbf{Languages:}\\
We should the basics of the languages like C++, javascript, python etc for manuplating the stylying and rendering of the map.\\\\
\textbf{Plots:}\\
Octave provides fltk as the default toolkit. But we can use gnuplot for more accurate plotting by setting them as default toolkit.\\\\
\textbf{Input:}\\
Input values are taken from user or default values defined in the file are used.\\\\
\textbf{Output:}\\
According to input values we will get the particular location of the map.

