\section{Software Requirement Analysis}

A Software Requirements Analysis for a software system is a complete 
description of the behavior of a system to be developed. It include functional Requirements
and Software Requirements. In addition to these, the SRS also contains 
non-functional requirements. Non-functional requirements are 
requirements which impose constraints on the design or implementation.
\begin{itemize}
\item{\bf Purpose}: Dynamic of structure is a web based software and the 
main purpose of this project is to:
\begin{enumerate}
\item Perform most of difficult Calculation work.
\item Make it work like batch mode. so, that user can give inputs 
together and relax.
\item Help M.Tech and Civil Engineer to analysis structure.
\item Automatic calculation of modal force and modes.
\item Reduce the time for analysis.
\item Provide on-line way to analysis so that individual does not have to 
install anything.
\end{enumerate}

\item{\bf Users of the System}
\begin{enumerate} 
\item Client : Clients are the end users that benefit from this software.
They just provide input and gets output in form of PDF.Client of this 
WEB Application can be of two types -:
\begin{enumerate}
\item Civil Engineer -: They have knowledge of working of procedure
and what input is being provided.
\item Layman -: They don't know anything about what's going on, their just 
work is to give input to system.   
 
\end{enumerate}
\end{enumerate}
\end{itemize}

\subsection{Functional Requirements}
\begin{itemize}
\item {\bf Specific Requirements}: This phase covers the whole requirements 
for the system. After understanding the system we need the input data 
to the system then we watch the output and determine whether the output 
from the system is according to our requirements or not. So what we have 
to input and then what we’ll get as output is given in this phase. This 
phase also describe the software and non-function requirements of the 
system.
\item {\bf Input Requirements of the System}
\begin{enumerate} 
\item Type of soil
\item Number of storeys
\item Importance Factor
\item Response Reduction Factor
\item Zone Factor
\item Input method (CSV or manual)
\item Output method (Email or direct PDF)
\item Mass of each storey
\item Height of each storey
\item Stiffness of each storey
\item Input in form of csv file
\end{enumerate}
\vskip 0.5cm
\item {\bf Output Requirements of the System}
\begin{enumerate} 
\item Calculation of modal force and modes.
\item Generation of output in form of PDF.
\item Mailing output PDF 
\end{enumerate}
\vskip 0.5cm
\item {\bf Special User Requirements}
\begin{enumerate} 
\item Automatic Email Generation of Output and Sending to the concerned person.
\item Taking bulk input values in form of CSV file
\end{enumerate}
\vskip 0.5cm
\item {\bf Software Requirements}
\begin{enumerate} 
\item Programming language: Python 2.7
\item software: SAGEMATH, \LaTeX{}
\item Framework: Django 1.7, Bootstrap
\item Web Languages: Html, Java Script, CSS
\item Database: Sqlite 
\item Documentation: Doxygen 1.8.3
\item Text Editor: Vim
\item Operating System: Ubuntu 12.04 or up
\item Revision System: Git

\end{enumerate}
\vskip 0.5cm
\subsection{Non functional requirements}
\begin{enumerate} 
\item Scalability: System should be able to handle a number of users. 
For e.g., handling around thousand users at the same time.
\item Usability: Simple user interfaces that a layman can understand.
\item Speed: Processing input should be done in reasonable time
 i.e. we can say maximum 24 hrs.
\end{enumerate}
\end{itemize}
